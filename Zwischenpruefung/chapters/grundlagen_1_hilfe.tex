\part{Grundlagen der Ersten Hilfe}
\chapter{Notruf absetzen}

\section{Notrufnummern}
\begin{itemize}
    \item 122 $\implies$ Feuerwehr
    \item 133 $\implies$ Polizei
    \item 144 $\implies$ Rettung
    \item 112 $\implies$ Euro-Notruf
    \begin{itemize}
        \item Funktioniert überall, wo es Handy-Empfang gibt; auch in Netzen von anderen Betreibern!
    \end{itemize}
    \item 01/406 43 43 $\implies$ Vergiftungsinformationszentrale
    \item 1450 $\implies$ Allgemeine Gesundheitsberatung (Corona...)
\end{itemize}

\section{Allgemeines}
\begin{itemize}
    \item Medizinische Notfälle $\implies$ Rettung
    \item Leitstelle gibt Anweisungen \& beendet den Anruf!
    \item Wichtigste Frage: \textbf{Wo?}
\end{itemize}

\chapter{Basismaßnahmen}
$\implies$ Können bei jedem Notfall durchgeführt werden!
\begin{enumerate}
    \item Person richtig lagern $\implies$ Muss für die Person angenehm sein!
    \item Für frische Luft sorgen ($\implies$ evtl. Fenster öffnen); beengende Kleidung öffnen!
    \item Person zudecken, wenn notwendig
    \begin{itemize}
        \item Die silberne Seite der Decke reflektiert Wärme $\implies$ Zum Wärmen nach Innen, zum Kühlen nach außen!
    \end{itemize}
    \item Person beruhigen, für sie da sein \& nicht von ihrer Seite weichen! (Psychische Betreuung)
\end{enumerate}

\chapter{Wegziehen / Umdrehen}

\section{Wegziehen}
\begin{itemize}
    \item Personen müssen aus der Gefahrenzone gebracht werden, wenn es die eigene Sicherheit erlaubt
\end{itemize}
\subsection*{Ablauf}
\begin{itemize}
    \item Person an den Schultern rütteln \& ansprechen
    \item Reagiert nicht $\implies$ Hände überkreuzen, Kopf darauf legen und wegziehen
\end{itemize}

\section{Umdrehen}
$\implies$ Person muss umgedreht werden, falls sie am Bauch liegt, damit Atmung kontrolliert werden kann!

\subsection*{Ablauf}
\begin{enumerate}
    \item Bewusstseinscheck - an den Schultern rütteln \& ansprechen (immer von der Seite annähern, von der die Person einen sehen kann)
    \begin{itemize}
        \item Falls nicht schon vor Wegziehen erfolgt
        \item Notruf absetzen / Um Hilfe rufen
    \end{itemize}
    \item Gesicht auf eine Seite drehen, falls nicht bereits in dieser Position vorgefunden
    \item Arme rotieren
    \begin{itemize}
        \item Blickrichtung $\implies$ Arm parallel an den Körper anlegen
        \item Entgegen der Blickrichtung $\implies$ Arm nach vorne ausstrecken
        \item Wichtig: Arme \textbf{nur} entlang des Bodens bewegen, um Verletzungen zu vermeiden
    \end{itemize}
    \item Person darüber informieren, dass sie jetzt umgedreht wird
    \item An Hüfte und Schulter packen und vorsichtig auf den Rücken drehen
    \item Atmung kontrollieren \& weitere Maßnahmen einleiten
\end{enumerate}

\subsection*{Mit Decke}
\begin{itemize}
    \item Decke zuerst in der Hälfte falten, danach wieder in der Hälfte zurückklappen \& neben der Person positionieren
\end{itemize}

\section{GAMS-Regel}
\begin{itemize}
    \item  G:  Gefahr erkennen (mit allen Sinnen)
    \item  A:  Abstand halten, absichern (Folgeunfälle vermeiden)
    \item (M): Menschenrettung durchführen, sofern gefahrlos möglich (\textbf{Eigenschutz vor Fremdschutz!})
    \item  S:  Spezialkräfte anfordern (Notruf\dots)
\end{itemize}

\section{Rettungskette}
\begin{itemize}
    \item Absichern (Warnweiste, Aussteigen)
    \item Notruf (144) / Erste Hilfe
    \item Rettung
    \item Krankenhaus
    \item Reha
\end{itemize}

\section{Zusammenfassung}
\begin{itemize}
    \item Etwa 75\% aller Unfälle passieren zu Hause, in der Freizeit oder beim Sport!
\end{itemize}