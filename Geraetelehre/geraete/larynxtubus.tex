\chapter{Handhabung des Larynxtubus}
\begin{enumerate}
    \item Generell erst ab der Pubertät bzw. für dieses Alter (14 Jahre) üblichen Proportionen einsetzbar
    \item Richtige Tubusgröße bestimmen
          \begin{itemize}
              \item \textbf{Gelb:} $<$ 155 cm
              \item \textbf{Rot:} 155 - 180 cm
              \item \textbf{Violett:} $>$ 180 cm
          \end{itemize}
    \item Tubus entblocken
    \item Tubus mit Gleitgel einschmieren
    \item Kinn mit Chin-Lift heben, Tubus einführen (bis letzte Markierung auf Höhe der Zähne ist)
    \item Spritze mit richtiger Luftmenge füllen und Cuffs blocken
    \item Probebeatmung durchführen
          \begin{itemize}
              \item Erfolgreich: Fixieren
              \item Nicht erfolgreich: Eine Markierung weiter hinaus ziehen, erneut probieren
              \item Noch immer nicht erfolgreich: Andere Tubusgröße ebenfalls 2x testen, sonst Guedeltubus
          \end{itemize}
    \item Fixieren
          \begin{itemize}
              \item Mullbinde halbieren, um den Kopf legen, 2x verknoten, auf die andere Seite des Tubus, Knoten machen $\implies$ Tubus sollte sich zum Ende in einem Mundwinkel befinden
              \item Adapter auf den Tubus stecken und mit Gummi fixieren
          \end{itemize}
\end{enumerate}