\part{Erkrankungen}

\chapter{Herzinfarkt}
\section*{Symptome}
\begin{itemize}
    \item Starke Schmerzen in der Brust
    \item Kaltschweißigkeit
    \item Angst
\end{itemize}

\section*{Lagerung}
\begin{itemize}
    \item Mit erhöhtem Oberkörper, damit Herz entlastet wird
\end{itemize}

\section*{Behandlung}
\begin{itemize}
    \item Rettung rufen
    \item Keine körperliche Anstrengung (Stiegen\dots)
    \item Person bricht zusammen $\implies$ Sofort mit der Wiederbelebung beginnen
\end{itemize}

\section*{Grund}
\begin{itemize}
    \item Herzkranzgefäße werden z.B. durch Blutgerinnsel verstopft, wodurch das Herz nicht mit ausreichend Blut versorgt wird \& das Gewebe abstirbt
\end{itemize}

\chapter{Schlaganfall}
\section*{Arten}
\begin{itemize}
    \item \textbf{Hirninfarkt:} Ein Blutgefäß innerhalb des Gehirns wird verschlossen (z.B. durch Gerinnsel) $\implies$ Häufiger!
    \item \textbf{Hirnblutung:} Ein Gefäß innerhlab des Gehirns platzt; Bluterguss drückt auf das Hirngewebe
\end{itemize}

\section*{Symptome}
\begin{itemize}
    \item FAST-Test:
    \begin{itemize}
        \item \textbf{F}ace: Lächeln möglich? (Typisch ist eine einseitige Lähmung, d.h. nur ein Mundwinkel reagiert)
        \item \textbf{A}rms: Können die Arme gehoben werden (Wieder einseitig\dots)
        \item \textbf{S}peech: Kann ein einfacher Satz nachgesprochen werden?
        \item \textbf{T}ime: Es muss sofort ein Notruf abgesetzt werden
    \end{itemize}
\end{itemize}

\section*{Lagerung}
\begin{itemize}
    \item Seitenlage (auf die angenehmere Seite); evtl. Tuch um Speichel aufzusaugen
\end{itemize}

\section*{Behandlung}
\begin{itemize}
    \item Rettung rufen
    \item Vergewissern, dass die Person weiterhin ansprechbar bleibt und gut atmen kann
\end{itemize}

\chapter{Krampfanfall}

\section*{Symptome}
\begin{itemize}
    \item Plötzlich einsetzende, starke Krämpfe am ganzen Körper; kann Symptom von Atem-Kreislaufstillstand sein
\end{itemize}

\section*{Lagerung}
\begin{itemize}
    \item Seitenlage, damit die Atmung freibleibt
\end{itemize}

\section*{Behandlung}
\begin{itemize}
    \item Rettung rufen
    \item Verletzungsrisiko minimieren $\implies$ Gegenstände aus dem Weg räumen
    \item Nach dem Anfall überprüfen, ob die Person normal atmet (sonst Wiederbelebung); richtig lagern
\end{itemize}

\chapter{Zuckerkrankheit}
\section*{Arten}
\begin{itemize}
    \item Typ-1-Diabetes:  Beginnt im Kindes-/Jugendalter, Produktion findet gar nicht statt
    \item Typ-2-Diabetes: Entsteht durch verminderte Empfindlichkeit der Zellen für Insulin; produzierende Zellen sind zudem erschöpft und können den erhöhten Bedarf nicht gewährleisten; entsteht oft durch Lebensweise
    \item Insulin: Senkt Blutzuckerspiegel
\end{itemize}

\section*{Symptome}
\begin{itemize}
    \item Diabetiker, der sich unwohl fühlt / eigenartig verhält
    \item Kalter Schweiß
    \item Blasse Gesichtsfarbe
    \item Kopfschmerzen
    \item Schneller Puls
\end{itemize}

\section*{Lagerung}
\begin{itemize}
    \item Beine hochlagern
\end{itemize}

\section*{Behandlung}
\begin{itemize}
    \item Rettung rufen
    \item Etwas Zuckerhaltiges zum Essen / zu trinken geben
\end{itemize}

\chapter{Asthmaanfall}
$\implies$ Atemwege verkrampfen

\section*{Symptome}
\begin{itemize}
    \item Pfeifende Ausatmung (Atemnot); Lippen werden blau
\end{itemize}

\section*{Lagerung}
\begin{itemize}
    \item Oberkörper nach oben, damit Atmung erleichtert wird (+ mit Armen abstützen $\implies$ Atemhilfsmuskulatur wird aktiviert)
\end{itemize}

\section*{Behandlung}
\begin{itemize}
    \item Rettung rufen
    \item Notfallmedikamente einnehmen
    \item Beengende Kleidung öffnen
    \item Person beruhigen (Lippen zusamenpressen $\implies$ Person atmet langsamer \& beruhigt sich); richtig lagern
\end{itemize}

\chapter{Kollaps}
$\implies$ Zu wenig Sauerstoff im Gehirn, Person klappt zusammen

\section*{Symptome}
\begin{itemize}
    \item Kurze Bewusstseinsstörung, die sich innerhalb von wenigen Sekunden wieder legt
\end{itemize}

\section*{Lagerung}
\begin{itemize}
    \item Beine hoch, damit der Kreislauf unterstützt wird
\end{itemize}

\section*{Behandlung}
\begin{itemize}
    \item Zustand bessert sich nicht sofort $\implies$ Rettung rufen
\end{itemize}

\chapter{Hitzenotfall}

\section*{Arten}
\begin{itemize}
    \item Sonnenstich: Der Kopf / das Gehirn war zu lange der prallen Sonne ausgesetzt
    \item Hitzschlag: Es ist allgemein zu warm, der Körper kann die Temperatur nicht mehr regulieren
\end{itemize}

\section*{Symptome}
\begin{itemize}
    \item hochroter Kopf
    \item Schwindel
    \item Übelkeit
    \item Kopfweh
\end{itemize}

\section*{Lagerung}
\subsection*{Sonnenstich}
\begin{itemize}
    \item Mit dem Kopf nach oben, damit Druck auf das Gehirn sinkt
\end{itemize}

\subsection*{Hitzschlag}
\begin{itemize}
    \item Mit den Beinen nach oben, damit Flüssigkeit zu den lebenswichtigen Organen gelangt
\end{itemize}

\section*{Behandlung}
\begin{itemize}
    \item Kühle Umschläge am Kopf / im Nacken
    \item Mit Trinken versorgen
    \item Arzt kontaktieren, falls keine Besserung eintritt
\end{itemize}

\chapter{Vergiftung}

\section*{Symptome}
\begin{itemize}
    \item Kopfschmerzen
    \item Übelkeit
    \item Erbrechen
    \item Schwindel
    \item Bewusstseinsstörungen
\end{itemize}

\section*{Lagerung}
\begin{itemize}
    \item Seitenlage, falls die Person erbricht\dots
\end{itemize}

\section*{Behandlung}
\begin{itemize}
    \item Etwaige Substanzen, die noch im Mund sind, ausspucken
    \item Rettung rufen
\end{itemize}

\chapter{Allergische Reaktion}
$\implies$ Überreaktion des Immunsystems
\section*{Symptome}
\begin{itemize}
    \item Rötungen
    \item Schwellungen
    \item Atemnot
\end{itemize}

\section*{Lagerung}
\begin{itemize}
    \item Mit erhöhtem Oberkörper (abstützen\dots), damit Atmung erleichtert wird
\end{itemize}

\section*{Behandlung}
\begin{itemize}
    \item Notfallmedikament in Reichweite bringen
    \item Rettung rufen
    \item Atemwege von außen kühlen
\end{itemize}