\part{Leben Retten}

\chapter{Einer Person helfen, die nicht reagiert}

\section{Allgemein}
\subsection*{Ablauf}
\begin{enumerate}
    \item Person laut ansprechen und an den Schultern schütteln (Person reagiert trotzdem nicht\dots)
    \item Sofort um Hilfe rufen; Notruf absetzen
    \item Atmung überprüfen
    \begin{itemize}
        \item Kopf überstrecken, um die Atemwege frei zu machen (Zunge versperrt sie dann nicht mehr)
        \item Sehen (Brustkorb anschauen), hören, fühlen ($\implies$ Mund)
        \item Normalerweise erfolgt alle \textbf{4-5} Sekunden Atemzug, maximal dürfen zwischen 2 Atemzügen \textbf{10} Sekunden liegen!
    \end{itemize}
\end{enumerate}

\section{Person atmet normal}

\subsection*{Ablauf}
\begin{enumerate}
    \item Person in die stabile Seitenlage bringen (Kopf überstrecken!)
    \item Rettung alarmieren
    \item Atmung im Minutentakt kontrollieren (Eine Hand am Bauch, die andere am Rücken), evtl. zudecken!
\end{enumerate}

\section{Person atmet nicht}
\subsection*{Ablauf}
\begin{enumerate}
    \item Rettung alarmieren, Defi organisieren, Verbandskasten organisieren
    \item Mit der Wiederbelebung beginnen (30:2)
    \begin{enumerate}
        \item Beim Beatmen die Nase zuhalten!
        \item Erst aufhören, wenn Person aufwacht, wieder normal atmet oder die Rettung übernimmt
    \end{enumerate}
\end{enumerate}

\subsection*{Mit Defibrillator}
\begin{itemize}
    \item Muss von 2. Person organisiert und bedient werden
    \begin{itemize}
        \item Während der Defi geholt wird, wird weiterhin 30:2 ausgeführt!
    \end{itemize}
    \item 2. Person klebt Defi wie abgebildet auf und folgt den Anweisungen des Gerätes
\end{itemize}
\subsubsection*{Funktionsweise}
\begin{itemize}
    \item Herz generiert Strom und bedient somit die Herzmuskeln
    \item Kammerflimmern $\implies$ unkontrollierte elektrische Aktivität, einzige Einsatzmöglichkeit für Defibrillator ("Entflimmerer")!
    \item Das Herz hört während dem Schock kurz zum Schlagen auf \& fängt danach (hoffentlich) wieder normal zum Schlagen an
\end{itemize}
\subsubsection*{Sicherheitshinweise}
\begin{itemize}
    \item Patient darf weder während Analysephase (Gefahr von Verfälschung der Ergebnisse) noch während der Schockphase berührt werden
    \item Vor Schock $\implies$ \textbf{"Achtung Schock!"}
\end{itemize}

\chapter{Einer Person helfen, die sich verschluckt hat}
Oft verschlucken sich Patienten während dem Essen, da der Kehlkopfdeckel zu lange offen ist. Beim Verschlucken kann es entweder zu einer leichten oder schweren Verlegung der Atemwege kommen.

\section{Leichte Verlegung der Atemwege}
\begin{itemize}
    \item Wird durch das Hinaufhusten gelöst
    \item Deshalb $\implies$ Betroffene Person zum Husten animieren
\end{itemize}

\section{Schwere Verlegung der Atemwege}
\begin{itemize}
    \item Zeigt sich meist dadurch, dass die Person nicht mehr atmen, sprechen oder husten kann $\implies$ Es besteht \textbf{Lebensgefahr}!
    \item Lösung (abwechselnd):
    \begin{enumerate}
        \item 5x auf den Rücken (zwischen die Schulterblätter) schlagen
        \item 5x Heimlich-Handgriff
    \end{enumerate}
    \item Die Maßnahmen sind so lange durchzuführen, bis Besserung eintritt / die Person bewusstlos wird
    \begin{itemize}
        \item Besserung tritt ein $\implies$ Person unbedingt im Krankenhaus auf innere Verletzungen (Blutungen) überprüfen lassen
        \item Besserung tritt nicht ein bzw. Person wird bewusstlos $\implies$ Sofort mit der Wiederbelebung (30:2, Defibrillator) beginnen, Notruf wählen
    \end{itemize}
\end{itemize}

\chapter{Einer Person helfen, die stark blutet}
Eine starke Blutung ist daran zu erkennen, dass viel Blut in kurzer Zeit schwallartig / spritzend austritt.
\section*{Ablauf}
\begin{enumerate}
    \item Falls vorhanden Handschuhe anziehen
    \item Betroffene Person soll die betroffene Extremität hochlagern (Füße auf jeden Fall hochlagern, auch wenn nicht betroffen, damit Herz genügend Blut erhält) und manuellen Druck auf die Wunde ausüben
    \item Der Ersthelfer alarmiert die Rettung und legt einen Druckverband wie folgt an:
    \begin{enumerate}
        \item Eine Wundauflage wird auf die Wunde gelegt; es wird weiterhin manueller Druck ausgeübt
        \item Ein Druckkörper, welcher die folgenden Eigenschaften erfüllen muss, wird auf die Wunde gelegt, es wird weiterhin manueller Druck ausgeübt
        \begin{itemize}
            \item Flexibel
            \item Saugfähig
            \item Größer als die Wunde
        \end{itemize}
        \item Wundauflage + Druckkörper werden mit einer 2. Mullbinde fixiert (jedes Mal, wenn man beim Wickeln auf der dem Druckkörper gegenüberliegenden Seite "ankommt", soll der Verband etwas gespannt werden)
        \item Die Mullbinde wird am Ende geteilt, um einen Knoten binden zu können
    \end{enumerate}
    \item Achtung: Druckverbände können nur auf Extremitäten angebracht werden, sonst muss manueller Druck (mit einer Wundauflage) ausgeübt werden
\end{enumerate}