\part{Verkehrsunfall}
\chapter{Bei einem Verkehrsunfall absichern}
Absichern ist enorm wichtig, um sich selbst zu schützen und Folgeunfälle zu vermeiden
\section*{Ablauf}
\begin{enumerate}
    \item Warnblinkanlage einschalten
    \item Warnweste \textbf{vor dem Aussteigen} anziehen
    \item Warndreieck in ausreichendem Abstand aufstellen
\end{enumerate}

\chapter{Bei einem Verkehrsunfall helfen}
Bevor weitere Maßnahmen gesetzt werden, muss die Unfallstelle auf jeden Fall wie oben beschrieben abgesichert werden sowie der Notruf gewählt werden.
Danach muss die Atmung der verletzten Personen kontrolliert werden (Notruf nicht vergessen!) - Motorradfahrern muss hierzu der Helm abgenommen werden, während Autofahrer aus ihrem Fahrzeug gezogen werden müssen.
Je nach Ergebnis werden dann weitere Maßnahmen gesetzt.

\section{Motorradfahrer}
Um die Atmung bei einem Motorradfahrer zu kontrollieren, muss der Helm abgenommen werden. Dabei wird wie folgt vorgegangen:
\subsection*{Ablauf}
\begin{enumerate}
    \item Bewusstseinscheck - Ansprechen\dots
    \item Gegebenenfalls wegziehen bzw. umdrehen; danach erneut Bewusstseinscheck!
    \item Kopf zwischen den Beinen fixieren, Visier öffnen, ggf. Brille abnehmen, erneut Bewusstseinscheck
    \item Der Helm wird wie folgt abgenommen:
    \begin{enumerate}
        \item Der Verschluss des Helms wird geöffnet
        \item Der Helm wird mit beiden Händen auseinander gezogen und soweit geneigt, bis die Nase zum Vorschein kommt
        \item Mit der einen Hand wird der Kopf von unten gestützt, mit der anderen wird der Helm bei der Kinnpartie gepackt
        \item Der Helm wird gleichmäßig nach hinten gezogen (obere Hand) bzw. geschoben (untere Hand)
    \end{enumerate}
    \item Die Atmung wird wie vorhin beschrieben kontrolliert, ausgehend davon werden weitere Maßnahmen gesetzt
\end{enumerate}

\section{Autofahrer}
Um die Atmung bei einem Autofahrer zu kontrollieren, muss dieser aus dem Fahrzeug gezogen werden. Dabei wird wie folgt vorgegangen:
\subsection*{Ablauf}
\begin{enumerate}
    \item Unfallstelle absichern
    \item Fahrer (bei geschlossener Tür) ansprechen
    \begin{itemize}
        \item Falls er bei Bewusstsein ist, sollte er zum Aussteigen animiert werden
    \end{itemize}
    \item Tür öffnen, Fahrer erneut ansprechen (Achtung Airbag!)
    \item Zündung ausschalten, Warnblinkanlage einschalten, Handbremse anziehen (Achtung Airbag, so tief wie möglich halten!)
    \item Füße befreien
    \item Person mit der längeren Hand halten, mit der kürzeren Hand den Gurt öffnen (damit man sich nicht so tief in das Fahrzeug lehnen muss)
    \item Mit Rautekgriff aus dem Fahrzeug bergen (wichtig: Daumen außen!)
\end{enumerate}